\documentclass[twocolumn]{jsarticle}
\usepackage[dvipdfmx]{graphicx}
\usepackage{kenzemi}
\usepackage{amsmath}
\usepackage{nidanfloat}

\begin{document}

\title{折り返し型ギルバート乗算回路の周波数特性劣化}
\author{小島 光}
\date{2023年10月2日}
\abstract{以前から検討している折り返し型ギルバート乗算回路について、非線形な動作をする乗算回路について小信号等価回路に手を加え解析を行った。またPMOSの各端子間にキャパシタを付加し、各回路について小信号解析を行った。}
\keyword{ギルバート乗算回路,小信号解析,非線形回路}
\maketitle

\section{はじめに}
以前設計した従来型との比較のために設計した折り返し型ギルバート乗算回路ではサイズの大きなPMOSを使用している。したがって、ゲート面積が増大し寄生容量が大きくなることが考えられる。今週はこの寄生容量について、具体的にどの部分が最も周波数特性劣化の原因として大きいのかを検討した。

\section{動作点の変動}
MOSFETは流れているバイアス電流に応じて動作点が決まり、小信号では動作点付近で動作するため線形回路として扱うことができる(図1)。この時、トランスコンダクタンス$g_{m}$はバイアス電流$I_{d}$を用いて



\begin{equation}
    g_{m}=2\sqrt{KI_{d}}
\end{equation}
と表される。\par
しかし、現在検討している乗算回路(図2)では

\end{document}